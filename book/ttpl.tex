% main.tex
% this file contains the main typesetting information for the
% tetra text book.  it also contains front and back matter
% toc, etc. but pulls the chapters from individual files

\documentclass[14pt,twoside,openany]{memoir}
\usepackage{hyperref}
\usepackage{wallpaper}
\usepackage{tabularx}
\usepackage{textcomp}
\usepackage{graphicx}
\usepackage[labelfont=bf, font=bf, justification=raggedright,
    singlelinecheck=false, font=sf]{caption}
\usepackage{geometry}
\usepackage{color}
\usepackage{sectsty}
\usepackage{alltt}
\usepackage{float}
\usepackage{tcolorbox}

% define all the important meta information
\title{The Tetra Programming Language}
\author{Ian Finlayson, Ph.D. \\ Department of Computer Science \\
    The University of Mary Washington}
\date{\today}
\newcommand{\tpledition}{1\textsuperscript{st} Edition}

% fonts
\usepackage[T1]{fontenc}
\usepackage[bitstream-charter]{mathdesign}
\usepackage{FiraSans}
\usepackage{FiraMono}

% source code colors
\definecolor{keywordcolor}{RGB}{0, 64, 136}
\definecolor{typecolor}{RGB}{178, 140, 16}
\definecolor{builtincolor}{RGB}{115, 55, 140}
\definecolor{commentcolor}{RGB}{26, 93, 17}
\definecolor{valuecolor}{RGB}{170, 0, 0}
\definecolor{linenocolor}{RGB}{70, 70, 70}

% color of the background used for output boxes
\definecolor{outputbgcolor}{RGB}{245, 245, 245}
\definecolor{inputcolor}{RGB}{26, 93, 17}

% set my custom chapter style (based on "ger")
\chapterstyle{default}
\renewcommand*{\chapnumfont}{\normalfont\HUGE\bfseries\sffamily}
\renewcommand*{\chaptitlefont}{\normalfont\HUGE\bfseries\sffamily}
\renewcommand{\printchaptername}{\normalfont\HUGE\bfseries\sffamily Chapter} 
\renewcommand*{\chapterheadstart}{
\mbox{}\\\mbox{}\rule[0pt]{\textwidth}{1.4pt}\par}
\setlength{\midchapskip}{20pt}
\renewcommand*{\printchaptertitle}[1]{\chaptitlefont #1
\\
\vspace*{-8pt}
\\\mbox{}\rule[5pt]{\textwidth}{1.4pt}}

% set up the page geometry, normal 8.5 by 11 with margins
\geometry{
  body={6.5in, 8.5in},
  left=1.0in,
  top=1.25in
}

% an environment used for output boxes
\newenvironment{outputbox}
{\begin{tcolorbox}[colback=outputbgcolor, arc=0pt]
 \begin{alltt}}
{\end{alltt}
 \end{tcolorbox}}

% set the section headings in sans font
\setsecnumformat{\sffamily\csname the#1\endcsname\quad}
\setsecheadstyle{\Large\bfseries\sffamily}

% get rid of the garish link colors
\hypersetup {
    colorlinks = false,
    hidelinks = true
}

% set the memoir page style
\pagestyle{plain}

% title page stuff
\makeatletter
\def\maketitle{%
  \null
  \ThisCenterWallPaper{1.75}{skyway.jpg}
  \thispagestyle{empty}%
  \vfill
  {\color{white} \begin{center}\leavevmode
    \normalfont
      {\raggedright {\fontsize{44pt}{44pt}\selectfont \sffamily \textbf{\@title}\par}%
       \vskip 12pt
       {\Huge \sffamily \textbf{\tpledition}\par}}%
    \vskip 4in
      {\raggedleft \Huge \sffamily \textbf{\@author}\par}%
  \end{center}}%
  \vfill
  \null
  \cleardoublepage
  }
\makeatother

% begin the main document
\begin{document}

% don't put in silly blank pages
\let\cleardoublepage\clearpage

% plop in the title page, defined above
\maketitle

% do all the front matter junk
\frontmatter

% copyright etc.
{
    \setlength{\parindent}{0pt}
    \setlength{\parskip}{12pt}

    % put all the stuff at the bottom of the page
    \null
    \vfill

    \textcopyright \the\year\ by Ian Finlayson

    \href{https://creativecommons.org/licenses/by-sa/4.0/}{\includegraphics{cc}}

    This work is licensed under a
    \href{https://creativecommons.org/licenses/by-sa/4.0/}{Creative Commons
    Attribution-ShareAlike 4.0 International License}.  Electronic versions
    of this book may be found freely online at \url{http://tetra-lang.org/book}.

    Should you have any questions, suggestions or questions regarding this text,
    please feel free to email the author at \href{mailto:finlayson@umw.edu}{\texttt{finlayson@umw.edu}}

    The cover image is a photograph of the Sunshine Skyway Bridge which spans
    the Tampa Bay in Florida.  The image is attributed to Taildragon at \\
    \url{http://en.wikipedia.org}
}
\pagebreak

% make the table of contents
\tableofcontents

% and now put in the actual chapters
\mainmatter

\chapter{Computers \& Programs}


\section{What is a Computer?}






\section{What is a Program?}





\section{What is Tetra?}




\section{First Programs}

\def \codelabel {ch1hello}
The first program we will look at simply prints ``Hello World!'' to the screen.
The code for this program is given in Listing \ref{\codelabel{}}.
\def \codecaption{Hello World}
\input{chapter1/hello.ttr.tex}

The output of this program can be seen below:

\input{chapter1/hello.ttr.out}



\def \codelabel {ch1input}
The next program we will look at does input as well.
It can be seen in Listing \ref{\codelabel{}}.
\def \codecaption{Asking for Input}
\input{chapter1/input.ttr.tex}


The output of this program is given below:

\input{chapter1/input.ttr.out}

\section{Summary}


\section{Exercises}


\chapter{Variables and Expressions}


\section{Using Variables}


\section{Expressions}


\section{Data Types}

\section{Summary}


\section{Exercises}


\chapter{Selection Statements}

\section{Making Decisions}



\section{If statements}



\section{Conditions}


\section{Boolean Variables}


\section{Boolean Expressions}

\section{Summary}


\section{Exercises}


\chapter{Looping}

\section{While Loops}


\section{For Loops}



\section{Using Ranges}


\section{Summary}


\section{Exercises}


\chapter{Parallelism}


\section{Parallel Statements}

\section{Parallel Pitfalls}

\section{Parallel For Loops}
\section{Summary}


\section{Exercises}


\chapter{Functions}

\section{Calling Functions}


\section{Writing New Functions}



\section{Passing Parameters}

\section{Summary}


\section{Exercises}


\chapter{List Processing}

\section{Using Lists}


\section{Looping on Lists}


\section{List Operations}

\section{Summary}


\section{Exercises}


\chapter{Synchronization}


\section{Lock Statements}


\section{Named Locks}


\section{Background Statements}


\section{Wait Statements}


\section{Summary}


\section{Exercises}


\chapter{More Data Types}

\section{Tuples}


\section{Dictionaries}


\section{Summary}


\section{Exercises}


\chapter{Object-Oriented Programming}

\section{Using Objects}


\section{Creating Classes}


\section{Initializing Objects}


\section{Using Methods}



\section{Summary}


\section{Exercises}








\end{document}

